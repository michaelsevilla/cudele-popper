\section{Introduction}

% What is the problem?
File system metadata services in HPC have scalability problems because
administrative tasks, like checkpointing~\cite{bent_plfs_2009} or scanning the
file system~\cite{zheng:pdsw2014-batchfs}, contend for the same directories and
inodes. Applications perform better with dedicated metadata
servers~\cite{sevilla:sc15-mantle, ren:sc2014-indexfs} but provisioning a
metadata server for every client is unreasonable. This problem is exacerbated
by current trends in HPC, where architectures are transitioning from complex
storage stacks with burst buffer, file system, object store, and tape tiers to
more simplified stacks with just a burst buffer and object
store~\cite{bent:login16-hpc-trends}; this puts more pressure on data access
because more requests end up hitting the same layer and latencies cannot be
hidden while data migrates across tiers.

% What is HPC doing?
To address this, developers are relaxing the consistency and durability
semantics in the file system because weaker guarantees are sufficient for their
applications. For example, many batch style jobs do not need the strong
consistency that the file system provides, so
BatchFS~\cite{zheng:pdsw2014-batchfs} and DeltaFS~\cite{zheng:pdsw2015-deltafs}
do more client-side processing and merge updates when the job is done. HPC
developers are turning to these non-POSIX IO solutions because their applications
are well-understood ({\it e.g.}, well-defined read/write phases,
synchronization only needed during certain phases, workflows describing
computation, etc.) and because these applications wreak havoc on file systems designed for
general-purpose workloads ({\it e.g.}, checkpoint-restart's N:N and N:1 create
patterns~\cite{bent_plfs_2009}).

\begin{figure}[tb]
\centering
\includegraphics[width=0.35\textwidth]{figures/subtree-policies1.png}
\caption{ Illustration of subtrees with different semantics co-existing in a
global namespace.  For performance, clients can relax consistency on their
subtree (HDFS) or even decouple the subtree and move it locally (BatchFS,
RAMDisk). Decoupled subtrees can further relax durability for even better
performance.  Clients that require stronger guarantees (POSIX IO) can still reside
in the same namespace.  }\label{fig:subtree-policies}
\end{figure}

% One example
One popular approach for relaxing consistency and durability is to ``decouple
the namespace", where clients lock the subtree they want exclusive access to as
a way to tell the file system that the subtree is important or may cause
resource contention in the near-future~\cite{grider:pdsw2015-marfs,
zheng:pdsw2015-deltafs, zheng:pdsw2014-batchfs, ren:sc2014-indexfs,
bent:slides-twotiers}. Then the file system can change its internal structure
to optimize performance. For example, the file system could enter a mode that
prevents other clients from interfering with the decoupled directory.  This
delayed merge ({\it i.e.} a form of eventual consistency) and relaxed
durability improves performance and scalability by avoiding the costs of remote
procedure calls (RPCs), synchronization, false sharing, and serialization.
While the performance benefits of decoupling the namespace are obvious,
applications that rely on the file system's guarantees must be deployed on an
entirely different system or re-written to coordinate strong
consistency/durability themselves.

% What did we do
To address this problem, we present an API and framework that lets developers
dynamically control the consistency and durability guarantees for subtrees in
the file system namespace.  Figure~\ref{fig:subtree-policies} shows a potential
setup in our proposed system where a single global namespace has subtrees for
applications optimized with techniques from different state-of-the-art
architectures.  The HDFS\footnote{HDFS itself is not directly evaluated in this
paper, although the semantics and their performance is explored
in~\S\ref{sec:use-case-1}.} subtree has weaker than strong consistency because
for performance it lets clients read files opened for
writing~\cite{hakimzadeh:dais14-hdfs-consistency}, which means that not all
updates are immediately seen by all servers; the BatchFS and RAMDisk subtrees
are decoupled from the global namespace and have similar consistency/durability
behavior to those systems; and the POSIX IO subtree retains the rigidity of
POSIX IO's strong consistency.  Subtrees without policies inherit the
consistency/durability semantics of the parent and future work will examine
embeddable or inheritable policies.

Our prototype system, Cudele, achieves this by exposing ``mechanisms" that
developers use to specify their preferred semantics.  Cudele supports 3 forms
of consistency (invisible, weak, and strong) and 3 degrees of durability (none,
local, and global) giving the user a wide range of policies and optimizations
that can be custom fit to an application. We make the following contributions:

\begin{enumerate}

  \item A framework and API for assigning consistency/durability policies 
  to subtrees in the file system namespace; this lets users navigate
  the trade-offs of different metadata protocols on the same storage system.

  \item This framework lets subtrees with different semantics co-exist in a
  global namespace. We show how this scales further and performs better than
  systems that use one strategy for the entire namespace .

  \item A prototype that lets users custom fit subtrees to applications
  dynamically. 

\end{enumerate}

The last contribution lays the groundwork for future work on our prototype. It
is better than the current practice of mounting different storage systems in a global
namespace because there is no need to provision dedicated storage clusters to
applications or move data between these systems.  For example, the results of a
Hadoop job do not need to be migrated into CephFS for other processing; instead
the user can change the semantics of the HDFS subtree into a CephFS subtree.
This may cause metadata/data movement to make things strongly consistent again
but this is superior to moving all data across file system boundaries. We
enable studies that adjust these semantics over {\it time and space}, where
subtrees can change their semantics and migrate around the cluster without ever
moving the data they reference.

% Results
The results in this paper lay the groundwork for such a system and confirm the
assertions of ``clean-state" research that decouple namespaces; specifically
that the technique drastically improves performance (104\(\times\) speed up).
We go a step further by quantifying the costs of merging updates (7\(\times\)
slow down) and maintaining durability (\(10\times\) slow down). In our
prototype, we get an 8\(\times\) speedup and can scale to twice as many clients
when we assign a more relaxed form of consistency and durability to a subtree
with a create-heavy workload.  We use Ceph as a prototyping platform because it
is used in cloud-based and data center systems and has a presence in
HPC~\cite{wang:pdsw13-ceph-hpc}. 

%In the remainder of the paper, Section~\ref{sec:posix-overheads} quantifies the
%cost of POSIX IO consistency and system-defined durability and
%Section~\ref{sec:methodology-decoupled-namespaces} presents the Cudele
%prototype and API. Section~\ref{sec:implementation} describes the Cudele
%mechanisms and shows how re-using internal subsystems results in an
%implementation of less than 500 lines of code. The evaluation in
%Section~\ref{sec:evaluation} quantifies the overheads and performance gains of
%explored and previously unexplored metadata designs.
%Section~\ref{sec:related-work} places Cudele in the context of other related
%work.

